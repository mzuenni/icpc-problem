\begin{slide}
	\begin{block}{Problem}
		Given a sequence of cards, how many cards need to be inserted at some other position to obtain a sequence where the cards are grouped by suit?
	\end{block}
	\pause


	\begin{block}{Solution}
		\begin{itemize}
			\item Observe that since we can reinsert a card at any position, we only have to count how many cards need to be removed. \pause
			\item For a fixed order of suits, we can calculate the minimum number of cards to be removed using dynamic programming.
		\end{itemize}
	\end{block}
\end{slide}

\begin{slide}
	\begin{block}{Solution}
		Dynamic programming for a fixed order of suits:
		\begin{itemize}
			\item Let \texttt{dp[i][s]} be the minimum number of cards to remove such that the remaining of the first $i$ cards are ordered correctly and the last card is of suit $s$. \pause
			\item If a card $i$ is already in the correct group, i.e. of suit $s$, we can leave it there. \\
			$\Rightarrow \texttt{dp[i][s]} = \min\{\texttt{dp[i - 1][s]}, \texttt{dp[i - 1][s - 1]}\}$ \pause
			\item If a card $i$ is not in the correct group, i.e. not of suit $s$, we have to remove it.
			$\Rightarrow \texttt{dp[i][s]} = \min\{\texttt{dp[i - 1][s]}, \texttt{dp[i - 1][s - 1]}\}+1$
		\end{itemize}

		\smallskip
		\pause
		There are $k!$ possibilities to arrange $k$ suits. We can try all.

		\pause
		\smallskip
		Running time: $\mathcal{O}(k! \cdot n \cdot k)$, which is fast enough for $k = 4$.

		\pause
		\smallskip
		This problem can also be solved in $\mathcal{O}(k \cdot 2^k \cdot n)$.
	\end{block}
\end{slide}
% 	\begin{block}{Solution}
% 		First, observe that we can select at most one $1$ since $1+1=2$ and $2$ is prime.\\
% 		\pause
% 		\smallskip

% 		Let's rephrase the problem in terms of graph theory:
% 		Given a graph with $n$ vertices where two vertices are connected iff their sum is prime, find a maximum independent set.\\
% 		\pause
% 		\smallskip
% 		Now observe the following properties:
% 		\begin{itemize}
% 			\item No two vertices sum up to $2$.
% 			\item All primes that some vertices sum up to are odd.
% 			\item The sum of two numbers is only odd if we sum up an even and an odd number.
% 			\pause
% 			\item[$\Rightarrow$] The given graph is \emph{bipartite}.
% 		\end{itemize}
% 	\end{block}
% \end{slide}

% \begin{slide}
% 	\begin{block}{Solution}
% 		\begin{itemize}
% 			\item The complement of a maximum independent set is a minimum vertex cover.
% 			\item For bipartite graphs, the minimum vertex cover and the matching have the same size.
% 			\pause
% 			\item[$\Rightarrow$] The answer is $n-|\text{matching}|$.
% 			\smallskip
% 			\pause
% 			\item A maximum matching can be found in many ways:
% 			\begin{itemize}
% 				\item Any flow algorithm
% 				\item Kuhn's algorithm
% 				\item Hopcroft Karp's algorithm
% 			\end{itemize}
% 		\end{itemize}
% 	\end{block}
